\paragraph{Simplicial complexes.} 
A \emph{simplicial complex} is a collection of finite sets closed under taking subsets.
We call a member of a simplicial complex $K$ a \emph{simplex} of \emph{dimension $p$} if it has cardinality $p+1$, and denote the set of all such $p$-simplices $K_p$.
A $p$-simplex has $p+1$ \emph{faces} of dimension $p-1$, namely the subsets omitting one element. We denote these $[v_0,\dotsc,\hat{v}_i,\dotsc, v_p]$ when omitting the $i$'th element.
If a simplex $\sigma$ is a face of $\tau$, we say that $\tau$ is a \emph{coface} of $\sigma$. While this definition is entirely combinatorial, there is a geometric interpretation, and it will make sense to refer to and think of $0$-simplices as \emph{vertices}, $1$-simplices as \emph{edges}, $2$-simplices as \emph{triangles}, $3$-simplices as \emph{tetrahedra}, and so forth (see Figure~\ref{fig:data2complex}, (b)).

Let $C^p(K)$ be the set of functions $K_p\to\RR$, with the obvious vector space structure. These \emph{$p$-cochains} will encode our data. Define the linear \emph{coboundary} maps $\delta^p:C^p(K)\to C^{p+1}(K)$ by
\begin{equation*}
\delta^p(f)([v_0,\dotsc,v_{p+1}]) = \sum_{i=0}^{p+1} (-1)^i f([v_0,\dotsc,\hat{v}_i,\dotsc,v_{p+1}]).
\end{equation*}
Observe that this definition can be thought of in geometric terms: the support of $\delta^p(f)$ is contained in the set of $(p+1)$-simplices that are cofaces of the $p$-simplices that make up the support of $f$.

\paragraph{Simplicial Laplacians.}
We are in this paper concerned with finite abstract simplicial complexes, although our method is applicable to a much broader setting, e.g.\ CW-complexes. In analogy with Hodge--de Rham theory~\cite{madsen1997calculus}, we define the \emph{degree-$i$ simplicial Laplacian} of a simplicial complex $K$ as the linear map $C^i(K)\to C^i(K)$ given by $\lap_i = \lapu_i + \lapd_i = \delta^{i\ast}\circ\delta^{i} + \delta^{i-1}\circ\delta^{i-1\ast}$, where $\delta^{i\ast}$ is the adjoint of the coboundary with respect to the inner product (typically the one making the indicator function basis orthonormal). In most practical applications, the coboundary can be represented as a sparse matrix $B_i$ and the Laplacians can be efficiently computed as $L_i=B_i\transpose B_{i}+B_{i-1}B_{i-1}\transpose$\footnote{Note that the Laplacians carry valuable topological information about the complex: The kernel of the $k$-Laplacian is isomorphic to the $k$-(co)homology of its associated simplicial complex~\cite{eckmann1944,horak2013spectra}.}. The matrices $L_0$ and $B_0$ are the classic graph Laplacian and incidence matrix.
