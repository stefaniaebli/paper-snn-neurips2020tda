\theoremstyle{definition}
\newtheorem{definition}{Definition}
\newtheorem{example}[definition]{Example}
\newtheorem{note}[definition]{Note}

\theoremstyle{plain}
\newtheorem{theorem}[definition]{Theorem}
\newtheorem{lemma}[definition]{Lemma}
\newtheorem{proposition}[definition]{Proposition}
\newtheorem{corollary}[definition]{Corollary}

\newcommand{\RR}{\ensuremath{\mathbb{R}}}
\newcommand{\ZZ}{\ensuremath{\mathbb{Z}}}
\newcommand{\RP}{\ensuremath{\mathbb{RP}}}
\newcommand{\lapu}{\ensuremath{\mathcal{L}^{\text{up}}}}
\newcommand{\lapd}{\ensuremath{\mathcal{L}^{\text{down}}}}
\newcommand{\lap}{\ensuremath{\mathcal{L}}}
\newcommand{\iso}{\ensuremath{\cong}}
\newcommand{\rank}{\ensuremath{\text{rank}}}
\newcommand{\inv}{\ensuremath{^{-1}}}


\newcommand{\ie}{}
\def\ie/{i.e.}
\newcommand{\eg}{}
\def\eg/{e.g.}
\newcommand{\etc}{}
\def\etc/{etc.}

\newcommand{\gardfix}[2]{{\color{red}#1}\fxfatal{#2}}
\newcommand{\fix}[2]{{\color{red}#1}\fxfatal{#2}}
\newcommand{\thmcite}[1]{\citeauthor{#1}, \citeyear{#1}~\cite{#1}}
\newcommand{\suchthat}{\ensuremath{\, \mid \,}}

\newcommand{\Reffig}[1]{Figure~\ref{fig:#1}}
\newcommand{\reffig}[1]{figure~\ref{fig:#1}}
\newcommand{\refeq}[1]{equation~\eqref{eq:#1}}
\newcommand{\refsec}[1]{section~\ref{sec:#1}}

\newcommand{\ip}[2]{\ensuremath{\left\langle #1 , #2 \right\rangle}}
\newcommand{\norm}[1]{\ensuremath{\lVert #1 \rVert}}

\newcommand{\placeholderfigure}{\begin{tikzpicture}
      \draw[help lines] (0, 0) grid (8, 4);
      \node () at (4,2) {PLACEHOLDER};
  \end{tikzpicture}
}

\DeclareMathOperator{\cut}{cut}
\DeclareMathOperator{\im}{im}
\DeclareMathOperator{\proj}{proj}
\DeclareMathOperator{\VR}{VR}


%---------For commenting--------
\usepackage{color}
\definecolor{darkorchid}{rgb}{0.6,0.196,0.8}
\newcommand{\todo}[1]{{\color{red}[[\textbf{TODO: }#1]]}}
\newcommand{\stefania}[1]{{\color{blue}[[\textbf{Stefania says: }#1]]}}
\newcommand{\gard}[1]{{\color{green}[[\textbf{Gard says: }#1]]}}

%--------For Referencing--------
\newcommand{\appendref}[1]{Section~\ref{append:#1}}
\newcommand{\figref}[1]{Figure~\ref{fig:#1}}
\newcommand{\exref}[1]{Example~\ref{ex:#1}}
\newcommand{\secref}[1]{Section~\ref{sec:#1}}
\newcommand{\Thmref}[1]{Theorem~\ref{thm:#1}}
\newcommand{\thmref}[1]{Theorem~\ref{thm:#1}}
\newcommand{\defref}[1]{Definition~\ref{def:#1}}

